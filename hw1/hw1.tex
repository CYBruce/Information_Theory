% Homework template for Information Theory and Statistical Learning
% by Xiangxiang Xu <xiangxiangxu.thu@gmail.com>
% LAST UPDATE: Oct 3, 2019
\documentclass[a4paper]{article}
\usepackage[T1]{fontenc}
\usepackage{amsmath, amssymb, amsthm}
% amsmath: equation*, amssymb: mathbb, amsthm: proof
\usepackage{moreenum}
\usepackage{mathtools}
\usepackage{url}
\usepackage{enumitem}
\usepackage{bm}
\usepackage{graphicx}
\usepackage{subcaption}
\usepackage{booktabs} % toprule
\usepackage[mathcal]{eucal}
\usepackage{dsfont}
\usepackage[numbered,framed]{matlab-prettifier}
\input{itdef}

\lstset{
  style              = Matlab-editor,
  captionpos         =b,
  basicstyle         = \mlttfamily,
  escapechar         = ",
  mlshowsectionrules = true,
}
\begin{document}
\courseheader



\newcounter{hwcnt}
\setcounter{hwcnt}{1} % set to the times of Homework

\begin{center}
  \underline{\bf Homework \thehwcnt} \\
\end{center}
\begin{flushleft}
  \textcolor{gray}{Chenyu Tian}\hfill
  \today
\end{flushleft}
\hrule

\vspace{2em}
\setlist[enumerate,1]{label=\thehwcnt.\arabic*.}
\setlist[enumerate,2]{label=(\alph*)}
\setlist[enumerate,3]{label=\roman*.}
\setlist[enumerate,4]{label=\greek*)}

\flushleft
\rule{\textwidth}{1pt}
\begin{itemize}
\item {\bf Acknowledgments: \/} 
  \textcolor{gray}{This template takes some materials from course CSE 547/Stat 548 of Washington University: \small{\url{https://courses.cs.washington.edu/courses/cse547/17sp/index.html}}.}

%   \textcolor{red}{If you refer to other materials in your homework, please list here.}
% \item {\bf Collaborators: \/}
%   \textcolor{gray}{I finish this template by myself.} \textcolor{red}{If you finish your homework all by yourself, make a similar statement. If you get help from others in finishing your homework, state like this:}
%   \textcolor{gray}{
%   \begin{itemize}
%   \item 1.2 (b) was solved with the help from \underline{\hspace{3em}}.
%   \item Discussion with \underline{\hspace{3em}} helped me finishing 1.3.
%   \end{itemize}
% }
\item  \emph{I certify that all solutions are entirely in my words and that I have not looked at another student's solutions. I have credited all external sources in this write up.}
  %\framebox[\linewidth]{\rule{0pt}{10pt}\textcolor{gray}{\large Your signature}}
\end{itemize}
\rule{\textwidth}{1pt}


\vspace{2em}

\begin{enumerate}
  \setlength{\itemsep}{3\parskip}

\item
  \begin{enumerate}
    \item 
    \begin{enumerate}
        \item 
        \begin{equation*}
            \begin{align*}
                \E[\E[\rvx|\rvy \rvz]|\rvy] = & \E((\sum_{i} x_i P(\rvx=x_i| \rvy \rvz))|\rvy)\\
                =&\sum_j (\sum_{i} x_i P(\rvx=x_i|\rvy,\rvz=z_j))P(\rvz=z_j) \\
                =&\sum_i x_i(\sum_{j}P(\rvx=x_i| \rvy,\rvz=z_j)P(\rvz=z_j)) \\
                =&\sum_i x_i P(\rvx=x_i| \rvy) \\
                =&\E(\rvx|\rvy)
            \end{align*}
        \end{equation*}
        \item
        \begin{equation*}
             g(\rvy)\E[\rvx|\rvy]=\sum_{i} g(\rvy)x_i\Pr(\rvx=x_i|\rvy)=\E[xg(\rvy)|\rvy]
        \end{equation*}
        
        \item
        \begin{equation*}
            \begin{align*}
            \E[\rvx\E[\rvx|\rvy]]=&\E\left[\rvx\sum_{i} x_i\Pr(\rvx=x_i|\rvy)\right]\\
            =&\sum_{j,k}(x_j\sum_{i} x_i\Pr(\rvx=x_i|\rvy=y_k))\Pr(\rvx=x_j,\rvy=y_k)\\
            =&\sum_{i,j,k} x_i x_j \Pr(\rvx=x_i|\rvy=y_k)\Pr(\rvx=x_j,\rvy=y_k)
        \end{align*}
        \end{equation*}
        \begin{equation*}
            \begin{align*}
        \E[(\E[\rvx|\rvy])^2]=&\E\left[(\sum_{i} x_i\Pr(\rvx=x_i|\rvy))^2\right]\\
        =&\sum_{k}(\sum_{i} x_i\Pr(\rvx=x_i|\rvy=y_k))^2 \Pr(\rvy=y_k)\\
        =&\sum_{i,j,k} x_i x_j\Pr(\rvx=x_i|\rvy=y_k)\Pr(\rvx=x_j|\rvy=y_k)\Pr(\rvy=y_k)\\
        =&\sum_{i,j,k} x_i x_j\Pr(\rvx=x_i|\rvy=y_k)\Pr(\rvx=x_j,\rvy=y_k)
        \end{align*}
        \end{equation*}
        Thus, $\E[(\E[\rvx|\rvy])^2]=\E[\rvx\E[\rvx|\rvy]]$.
        
        \item
        \begin{equation*}
            \begin{align*}
                & \mathbb{E}[\operatorname{Var}(\mathrm{x} \mid \mathrm{y})]+\operatorname{Var}(\mathbb{E}[\mathrm{x} \mid \mathrm{y}])\\
                =& \E[\E[\rvx^2|\rvy] - (\E[\rvx|\rvy])^2] + \E[(\E[\rvx|\rvy])^2] - (\E[\E[\rvx|\rvy]])^2\\
                =& \E[\E[\rvx^2|\rvy]] - (\E[\E[\rvx|\rvy]])^2\\
                =& \E[\rvx^2] - (\E[\rvx])^2\\
                =& \operatorname{Var}(\mathrm{x})
            \end{align*}
        \end{equation*}
        Thus, $\operatorname{Var}(\mathrm{x})=\mathbb{E}[\operatorname{Var}(\mathrm{x} \mid \mathrm{y})]+\operatorname{Var}(\mathbb{E}[\mathrm{x} \mid \mathrm{y}])$.
    \end{enumerate}
    \item
    % 1.(b)
    \begin{enumerate}
        \item 
        \begin{equation*}
            \begin{align*}
        \cov(\underline{x}) = & \E[(\underline{x}-\E[\underline{x}])(\underline{x}-\E[\underline{x}])^\T] \\    
                    = & \E[\underline{x}\underline{x}^\T]-\E[\underline{x}]\E[\underline{x}]^\T\\
                    = & \E[\E[\underline{x}\underline{x}^\T|\rvy]]-\E[\E[\underline{x}|\rvy]]\E[\E[\underline{x}|\rvy]]^\T\\
                    = & \E[\Cov[\underline{x}|\rvy]+\E[\underline{x}|\rvy]\E[\underline{x}|\rvy]^\T]-\E[\E[\underline{x}|\rvy]]\E[\E[\underline{x}|\rvy]]^\T\\
                    = & \E[\Cov[\underline{x}|\rvy]]+\E[\E[\underline{x}|\rvy]\E[\underline{x}|\rvy]^\T]-\E[\E[\underline{x}|\rvy]]\E[\E[\underline{x}|\rvy]]^\T\\
                    = & \E[\Cov[\underline{x}|\rvy]]+\Cov[\E[\underline{x}|\rvy]]\\       
        \end{align*}
        \end{equation*}
        
        \item
        If $\exists \underline{c} \in \mathbb{R}^{k}, c \neq \underline{0}$, $\underline{c}^\T \underline{x} =d$ and $d$ is a constant. $\underline{c}^\T\E[\underline{x}]=d$ Then $\underline{c}^\T \underline{x}\underline{x}^\T\underline{c} =d^2$. Also, $\underline{c}^\T \E[\underline{x}\underline{x}^\T]\underline{c} =d^2$.
        
        \begin{equation*}
            \begin{align*}
        \underline{c}^\T \cov(\underline{x})\underline{c} = & \underline{c}^\T\E[\underline{x}\underline{x}^\T]\underline{c}-\underline{c}^\T\E[\underline{x}]\E[\underline{x}]^\T\underline{c}\\
        = & \underline{c}^\T\E[\underline{x}\underline{x}^\T]\underline{c}-\underline{c}^\T\E[\underline{x}]\E[\underline{x}]^\T\underline{c}\\
        = & 0
        \end{align*}
        \end{equation*}
        And,
        \begin{equation*}
            \begin{align*}
        \underline{c}^\T \cov(\underline{x})\underline{c} = & \underline{c}^\T\E[(\underline{x}-E[\underline{x}])(\underline{x}-E[\underline{x}])^\T]\underline{c}\\
        =&\E[\underline{c}^\T(\underline{x}-E[\underline{x}])(\underline{x}-E[\underline{x}])^\T\underline{c}]\\
        = & 0 \Longleftrightarrow (\underline{x}-E[\underline{x}])(\underline{x}-E[\underline{x}])^\T\underline{c} = \underline{0}
        \end{align*}
        \end{equation*}
        It has $\operatorname{cov}(\underline{x})c=\underline{0}$. 
        Thus, $\operatorname{det}(\operatorname{cov}(\underline{x}))=0$.
        
    \end{enumerate}
  
    \end{enumerate}
  \item
  \begin{enumerate}
      % 1.2 a
      \item 
      \begin{equation*}
      \E[(y-ax-b)^2]==a^2\E[x^2]+\E[y^2]+b^2-2a\E[xy]-2b\E[y]+2ab\E[x]
      \end{equation*}
      Take the derivatives with respect to $a$ and $b$ to find the minimum when the derivatives equal 0. It has
      \begin{equation*}
          \begin{align*}
              a^{*}\E[x^2]+b^{*}\E[x]=&\E[xy]\\
              a^{*}\E[x]+b^{*}=&\E[y]\\
          \end{align*}
      \end{equation*}
      Considering $\var(x)=\var(y)=\sqrt{\var(x)\var(y)}$, it has $a^{*}=\frac{\E[xy]-\E[x]\E[y]}{\var(x)}=\rho(x,y)$.
      % 1.2 b
      \item
      If $x \perp y$, 
      \begin{equation*}
          \begin{align*}
          \E[f(x)g(y)]=&\sum_{i,j}f(x_i)g(y_j)\Pr(x=x_i,y=y_j)\\
          =&\sum_{i,j}f(x_i)g(y_j)\Pr(x=x_i)\Pr(y=y_j)\\
          =&\left(\sum_{i}f(x_i)\Pr(x=x_i)\right)\left(\sum_{i}g(y_j)\Pr(y=x_j)\right)\\
          =&\E[f(x)]\E[g(y)]
          \end{align*}
      \end{equation*}
      \begin{equation*}
          \begin{align*}
          \rho(g(y), g(y))=&\frac{\mathbb{E}[(f(x)-\mathbb{E}[f(x)])(g(y)-\mathbb{E}[g(y)])]}{\sqrt{\operatorname{var}(f(x)) \operatorname{var}(g(y))}}\\
          =&\frac{\mathbb{E}[f(x)g(y)] - \mathbb{E}[f(x)] \mathbb{E}[g(y)]}{\sqrt{\operatorname{var}(f(x)) \operatorname{var}(g(y))}}\\
          =&0
          \end{align*}
      \end{equation*}
      So, $\forall f,g,\rho(f(x),g(y))=0$.
      
      On the other hand, if $\forall f,g,\rho(f(x),g(y))=0$. Let $f(x)=x,g(y)=y$. It exists
      \begin{equation*}
      \mathbb{E}[xy]=\mathbb{E}[x]\mathbb{E}[y] \Longleftrightarrow \Pr(x,y)=\Pr(x)\Pr(y)
      \end{equation*}
      So, $x \perp y$.
  \end{enumerate}
  % 1.3
  \item Denote $y=exp(x)$. Because exp(x) is monotonically increasing, it has
  \begin{equation*}
      \begin{aligned} f_{Y}(y) &=\frac{\mathrm{d}}{\mathrm{d} y} \operatorname{Pr}(Y \leq y)=\frac{\mathrm{d}}{\mathrm{d} y} \operatorname{Pr}(\ln Y \leq \ln y)=\frac{\mathrm{d}}{\mathrm{d} y} \Phi\left(\ln y\right) 
      \\ &=\frac{1}{\sqrt{2 \pi}} \exp \left(-\frac{(\ln y)^{2}}{2}\right) \frac{\mathrm{d} \ln y}{\mathrm{d} y}
      \\ &=\frac{1}{y} \cdot \frac{1}{\sqrt{2 \pi}} \exp \left(-\frac{(\ln y)^{2}}{2}\right) \end{aligned}
  \end{equation*}

%   \item You may need to add figures and source codes in your homework. Figure \ref{fig:1} is an example that compares the empirical distribution (histogram) and probability density function of a Gaussian random variable.
%     \begin{figure}[htbp]
%       \centering
%       \includegraphics[width = 0.8\textwidth]{pdf_normal.eps}
%       \caption{Gaussian PDF and histogram of samples}
%       \label{fig:1}
%     \end{figure}

%   The source code to plot Figure \ref{fig:1} could be found in Appendix \ref{sec:a:code}. Here are the core codes:
%   \lstinputlisting[firstline=4,lastline=4, firstnumber=4]{matlabscript.m}
%   \lstinputlisting[firstline=6,lastline=7, firstnumber=6]{matlabscript.m}
%   To understand line 6, note that if we have $n$ samples of $X$ denoted by $x^{(i)}, i = 1, 2, \cdots, n$, then the probability density function $p_{X}$ can be estimated as
%   \begin{equation*}
%     \begin{aligned}
%       p_{X}(x_0) &= \left.\frac{\mathrm{d}}{\mathrm{d}x} Pob(X \leq x) \right|_{x = x_0} \\
%       &\approx \frac{Pob(x_0 - \Delta x < X \leq x_0)}{\Delta x}\\
%       &\approx \frac{1}{n\Delta x} \sum_{i = 1}^n \1_{x^{(i)} \in (x_0 - \Delta x, x_0]}.
%     \end{aligned}    
%   \end{equation*}
    
\end{enumerate}
  
%   \newpage
  
%   \appendix
%   \section{Source code}
%   \label{sec:a:code}
%   % \lstlistoflistings
%   The source code for plotting Figure \ref{fig:1} is shown as follows.
%   \lstinputlisting{matlabscript.m}
  
\end{document}
%%% Local Variables:
%%% mode: latex
%%% TeX-master: t
%%% End:
